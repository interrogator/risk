%!TEX root = ../risk_report.tex

\chapter{A sociological conceptualisation of risk}


\section{History of risk research}


\noindent Since the 1980s and 1990s the notion of risk has become increasingly influential in societal discourses and scholarly debate \cite{skolbekken_risk_1995}. 

\section{Contemporary theories and methods}

From early work on risk and culture \cite{douglas_risk_1986,douglas_risk_2013} to the \emph{risk society} thesis \cite{beck_risk_1992,beck_world_2009,giddens_runaway_2002}, from governmentality theorists working in the tradition of Foucault \cite{dean_governmentality:_2010,omalley_risk_2012,rose_powers_1999} to modern systems theory \cite{luhmann_ecological_1989,luhmann_communication_1993} all have built their work around the notion of risk and implicitly or explicitly refer to linguistic changes. 

\section{Lack of data-driven approaches}

However, none of these approaches provides a detailed account of linguistic changes in recent history. Although Beck assumes an increase in risk debates after World War Two, he does not support his claims with detailed empirical analysis. Luhmann and Giddens provide anecdotal historical evidence of the shift towards risk, but deliver no detailed account of the linguistic dynamics after World War Two. In contrast, linguists have provided lexicographic descriptions of risk utilising corpus linguistic methods \cite{hamilton_meanings_2007} and have sketched out a cognitive-semantic frame for risk within frame semantics theory \cite<e.g.>{fillmore_toward_1992}. Lacking thus far, however, has been a functional linguistic account of longitudinal changes in the way risk is instantiated in written texts.

\section{Lack of longitudinal studies}

\begin{itemize}
    \item Of the few data-driven studies, none has been longitudinal.
\end{itemize}

\section{Risk in the age of Big Data}

\begin{itemize}
\item The web has made possible the analysis of quantities of data that until recently have been unimaginable. This adds new epistemologies of risk, and the ability to test claims made about risk within sociology. 
\item Through a chain of operationalisation, we can reduce key claims made by Beck and Giddens into linguistic realisation, and then into search patterns. Though it is important to remember that our investigation is of only one text type (print journalism), given the vast size of our dataset, it is sensible to assume that any changes in the behaviour of risk language could be located within our corpus.
\item Methods for building and interrogating Big Datasets are only emerging. In an even greater stage of infancy are methods for interpreting the results of Big Data interrogations, especially within the traditions of the humanities and social sciences.
\end{itemize}



%\bibliography{../../references/libwin}