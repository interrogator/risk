%!TEX root = ../risk_report.tex

\chapter{Theoretical foundations}

In this chapter, we discuss key tenets of sociological risk theory, and their relational to linguistic accounts of risk.

\section{Central hypotheses in risk studies}

\subsection*{From calculative technology to uncertain potentiality}

Governmentality considers risk as a calculative technology which is used to manage potential harm. Similarly in the risk society perspective insurance and science are characterised by risk calculation to minimise risk. In the risk society perspective the calculability of risk characterises early modern experience (M. Weber: definition of rationalization\slash modernization). If risks are not directly controllable by science\slash knowledge we still have the opportunity to manage them by insurance or example. However, both governmentality theorists as such as risk society researchers have emphasised that uncertainties and non-knowledge would increase and we would observe a shift from the calculability of risk to the potentiality of harm. If this is correct it is more likely to find phrases which indicate the calculability of risk compared to the pure potentiality of risk. 

\subsection*{From positive risk taking to exposure to risk}

The risk literature about societal changes has also emphasised that the experience of risk has started to change during modernisation on another dimension. The positive side of risk as risk-taking would lose influence (Douglas, Lupton). Risk would mainly mean harm or danger. If this is correct we would expect an overall clear and significant decrease in verbal risk forms which indicate an active engagement into risk for a gain. As a result we would expect a decrease in verbal forms involving an active decision to take a risk for something positive. We might even observe within the verbal forms a shift from positive risk taking to a pure exposure to risk where the possible gain disappears. For example, the notion of taking a risk or running a risk might increasingly be supplanted by notions of exposure to risk.

\subsection*{Expected risk taking but lacking control}

Governmentality theorists have claimed in recent decades that a neo-liberal agenda has become more dominant that shifts responsibility to individuals and the expectation that individuals actively make decisions and take risks. If this is correct we would expect more individualised phrases which express more active risk taking. 

However, Beck claimed that in recent decades one has to understand and act as an individualised planning office exactly at times where knowledge and control of the future is limited. That means an active risk taking citizen is expected at a time where it is even more unlikely that an individual can control outcomes. For such a contradictory situation we would expect less the communication of self-confident decision making and risk taking but an individualised suffering of all kinds of risk. 

This would support the suggestion of social policy researchers that risk is increasingly shifted from organisations and institutions to individuals (Hacker: risk shift). It is important to see that this happens as a legitimate shift not something what happens against public resistance. At least in a country such as the US where individual action is highly valued we would expect that this shift takes place legitimately and deeply rooted in the societal institutions. Rather than as a surprise it would be a consequent development following an already prepared path. As a result we would expect not only as a rational of consequent media reporting that individual stories are but to sell to the public but that more generally the individual exposure to risk rather than individual agency would be emphasised. 

\subsection*{The increasing salience of at-risk status in risk reporting}

Ulrich Beck claimed in the chapter Beyond Class and Status in his famous book the Risk Society that social inequalities and disadvantage would increasingly be framed in individualised terms. That means that risk is no longer attributed to social class or status but to social groups which are at-risk because of their particular behaviour rather than class affiliation. 
This would be supported by claims of researchers examining shifts in public\slash social policy and social work claiming that social institutions would increasingly use practices that identify social groups at-risk on the basis of particular indicators which then characterise particular groups such as drug users, homeless, fatherlessness etc. as at-risk groups which require regulation, support, encouragement or protection. If Kemshall and others are correct that risk thinking has become a common societal practice this should be reflected in media coverage.

We would expect that groups reported on in the media are identified and reported about using their at-risk status rather than social class affiliation or general socio-structural conditions which influence their behaviour or shape their living conditions. Such generalised factors would be rather silenced or made invisible. We would expect that it is increasingly likely that we find groups characterised by attributed risk status.

\subsection*{Individualisation winners and losers---risk-takers versus at-risk groups}

There is also a tension in the debates about risk in the literature. Relative powerful middle class people are assumed to be individualisation winners, that means they have agency and can make decisions while more disadvantaged people lack agency and are approached by the state, encouraged or more broadly managed. They have a more intrinsic quality of being at-risk. For example, drug users might be a population at-risk by social definition. We would expect finding a clear distinction between powerful risk takers and powerless at-risk or vulnerable groups identified and characterised by a specific variable or characteristic.

%Interesting to see which groups are represented in the NYT. Obviously powerful politicans. But also on a lower level, individual investors. Where are patients positioned? Differences? At risk or risk takers?

\section{Our research approach}

The social sense-making processes of risk depends on risks being communicated. Though people experience risk when they manifest personally, since risks are usually expectations towards the future, the social process that shape these expectations are crucial. Even when we make personal experiences it depends on broader social processes whether we interpret a hot summer as an indication for climate warming or just a normal variation.

Communication is mediated through language, and language is by no means restricted to a neutral communication of knowledge or information about events and happenings in the world. Language may shape what seems possible as much as what seems appropriate or inappropriate. It both constructs and responds to all kinds of information about the context in which it has been generated, the values underpinning it, the power structures it reproduces or is structured by (sociolects; gendered language, etc.). Since language is such a rich resource for communicating information about social reality, it is also data that can be used to examine social change (e.g. Norbert Elias' historical analysis of the books of manners to examine the civilisation process). 

The media plays an important role in communicating social life. It not only influences but also reflects what is considered important at a historical point in time not only in the form of selecting particular content but how it is presented. A careful analysis of linguistic change therefore requires not only investigation of what has been communicated through language, but how it has been communicated and how both lexis and grammar have changed over time in the communication of issues such as risk.

Sociology, linguistics and media studies provide slightly different concepts of both `context' and of the forces that influence the selection and communication of social issues such as risk. Sociology is interested in wider and long term social changes. In a historical perspective, the focus is on how institutional and sociocultural social changes are reflected in the use of language. Sociologists are well aware of that the use of language is, for example, influenced of the social milieu a person is part of (e.g. working class, middle class) and such a context manifests not only in the content but also the form and the use of grammar of language. For sociologists, contexts and events within contexts are not necessarily socially triggered or caused. But how they are dealt with is mediated through language. The suppression of women in a society might be openly debated or not talked about. It might even be engraved in a language, where masculine nouns and\slash or pronouns have historically also been used to refer to general populations (e.g. \emph{A giant leap for mankind}) or singular entities whose gender is unknown.

% with regard to the above, of course linguists are attuned to class, etc, and naturally, moreso, when what is being considered is dialectal variation.

In many branches of functional linguistics, the understanding of context focusses on text. A particular text can be analysed regarding its form and structure and its origin. Through linguistic features of texts alone, genre can often be clearly determined. Whether a text is a newspaper article or a university lecture, a talk of a party leader to party members or a general public, can often be determined simply through an analysis of the lexis and grammar in a text, as well as the way in which stages of the text are ordered. The larger social conditions and how these might have influenced the content and use of language are less commonly examined. Despite increasing awareness and sensitivity to context in functional linguistics, context is more commonly operationalised as observable constellations of variables of a given interaction (speaker demographics, spoken\slash written, formality, etc), rather than as a set of broader social movements, ideas and values. Even researchers within systemic functional linguistics (SFL), which at one time explicitly attempted to delineate the relationship between realised language and social class and ideology, have revised the conceptualisation of context to exclude ideology as the greatest level of observable abstraction. Long-term historical analyses remain centred on language, and empirically driven attempts to connect language change to broader social change are exceptionally rare.

This is not to say that there is no value of linguistic theory and methods for the purposes of understanding the changing status of risk in society. In fact, the opposite is the case: linguistics (in our case, SFL) provides a framework for delineating the \emph{kinds} of changes that risk language undergoes. For example, in order to understand how risk language has changed, we must first distinguish between risk as a participant within a communication about the world (\emph{The risk was serious}) and risk as a process (\emph{Lives were risked}). Our addition to more standard linguistic methods is not that we abstract the significance of linguistic changes---as this is a common task within linguistic discourse analysis---but rather that following from an abstracted discourse-semantic analysis of risk, we abstract again, to consider the influence of factors beyond what is captured within linguistic taxonomies of context.

Media studies are positioned in between sociological and linguistic approaches. Discourse analyses using media or print media often focus on content and the positive or negative representation of issues. These studies do often not go into further detail regarding long term linguistic changes. They tend to focus on short term ways of representation of issues such as climate change. However, media studies have also raised awareness of the organisational and social context that shapes how news are produced (e.g. free press or more or less controlled press; economic pressure; political bias). Research has examined the production process of news and how this process follows an own logic of newsworthiness that influences which issues enter the media and which not. There is also awareness that there are events and dimensions of change which are not reported in the media. Not everything is newsworthy and what is selected follows the own media production logic of news. In this respect media reporting is selective and it is difficult take stock of the aspects which have not been reported without looking beyond the media. These issues must be identified and approached differently. For example, it is important for linguistic research of texts alone to acknowledge that such approaches may not be able to consider what drives the media agenda and which kinds of texts might be systematically included\slash excluded as a result of unobserved institutional and contextual factors.

% point conceded, above.

However, since the media are part of social change, it reflects as much as influences social changes, and, accordingly, can be used to examine long term social change. Since many risk issues are newsworthy, we can expect to find a lot risk communication, which allows us to examine the changing practice of risk reporting and the use of the risk semantic. Broad changes in the relationship between news institutions and risk communication (e.g. which risks are considered, how they are reported, etc.) are so general and part of more generally changing discourses and linguistic practice that they will affect newspapers as well since they have to appeal to the public.

\subsection{Summary}

We adopt an interdisciplinary approach to risk research. For this reason, a key component of our project involved developing and applying a novel methodology that combines systemic functional linguistics, corpus linguistics, and sociological theory concerning risk. In the following chapters, we outline relevant linguistic theory, drawing mostly on corpus, computational and systemic-functional theory. Chapters 5--7 present and discuss the findings of the investigation, and Chapter 8 provides a research agenda and summary.


