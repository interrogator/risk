%!TEX root = ../risk_book.tex

\chapter{Discourse-semantics of \emph{risk} in the NYT}

	Accordingly to SFL, the sum total of lexicogrammar, abstracted, realises the discourse-semantics of texts. Accounting for discourse-semantic meaning involves sensitivity to realised lexicogrammatical forms, but also to the ways in which incongruence and grammatical metaphor can create similar meanings through differing grammatical constructions: as noted earlier, \emph{potential harms} may be realised as a participant in a process of risk (\emph{Bush risked losing the election}), or as a modifier of a risk participant (\emph{the cancer risk\slash the risk of cancer}).\endnote{A key issue in CADS is the ability to systematically account for rank-shifted meanings (See McDonald, Forthcoming).}~Given the diversity of roles in which risk words can appear, the delineation of risk by roles within mood and transitivity systems in the previous section was thus a methodological necessity, but one with heavy ramifications for analysis. At the level of discourse-semantics, it becomes necessary to discuss risk word behaviour more fluidly, with reference to both experiential and interpersonal meanings, and with distinctions between risk as participant, process and modifier largely collapsed. This is perhaps especially so in in our case, as risk is an example of a lexical item that may be congruently realised as either participant and process, straddling the semantic space between entity and event.

	The first part of this discussion provides a description of risk in the NYT absent longitudinal considerations---something akin to the descriptions provided by \citeA{hamilton_meanings_2007} and \citeA{fillmore_toward_1992}, but from a systemic-functional, rather than frame-semantic purview. The second part is concerned with accounting for shifting discourse-semantics of risk, via the lexicogrammatical findings presented in the previous section. In the final section, longitudinal shifts are discussed in the context of specific events, broader social change, and sociological theory.

	%as well as a brief commentary on the usefulness of sociological perspectives in tandem with SFL as a means of understanding the relationship between text, discourse, language and context.

	\section{A monochronic description of risk}

		Before turning our attention to the behaviour of risk words over time, it is useful to provide a short description of the way risk words are generally used in the NYT.

		Foremost, striking is the ability of risk to function within all open word classes (noun, adjective, verb, adverb), as well as the sheer diversity of risk words. 507 unique lexical items containing risk were found\endnote{This naturally depends on your definition of a word\slash token. If we removed hyphenates or tokens containing a slash (\emph{risk\slash reward}), the list would be dramatically reduced in size. Lemmatisation would compress this list even more.}~, including many (albeit vary rare) words lacking existing lexicographical description: examples such as \emph{risk-shy}, \emph{risk-addicted}, \emph{risk-elimination}, \emph{species-at-risk} and \emph{risk-happy} demonstrate the overall salience of risk and the nuance with which it is instantiated in news discourse. Further testament to this salience are the nuanced distinctions in riskers' awareness of potential harm in \emph{risking}, \emph{putting at risk, }\emph{taking} and \emph{running} risks.

                        In many respects, our findings agree with those of other monochronic descriptions of risk language. First, we can see the usefulness of the frame-semantic categorisation of the kinds of participants\slash social actors that occur within the risk frame \cite<i.e.>{fillmore_toward_1992}: we often found it useful to divide corpus interrogation results into categories of \emph{risker}, \emph{potential harm}, \emph{risked thing}, and the like. Promising is the fact that in many cases, we can use the grammatical structure of the clause to automatically return lists of each kind of participant. In cases where the grammar alone cannot tell us the participant role (\emph{I risked my death}, \emph{I risked my life}), manual sorting is not difficult, as there is little ambiguity. If we insert the \emph{losing} participle (\emph{I risk losing my life}, but *\emph{I risk losing my death}), we can quickly determine if a result is a \emph{potential harm} or a \emph{risked thing}. This is especially so when risk is the \emph{process}, rather than a participant or modifier. With this in mind, focussing more exclusively on risk as process in very large parsed datasets may prove elucidating.

                        Our findings also agree with a key claim made by \citeA{hamilton_meanings_2007}: health and illness risks were surprisingly prominent within our data. As will be discussed below, however, this does not appear to be a purely static phenomenon: our longitudinal analysis points toward health risks as being far more common in contemporary language than in the language of our 1963 dataset.

                        A second point on which we agree is with their contention that risk words behave differently in different registers and genres, and that comparison of genres is worthy of further study (though here we rely on not on our dataset but on a long history of research in support of this contention within SFL):

                        \begin{quote}
                        We find in these discourse environments that the focus of the semantic prosody and the semantic preference changes according to the context in which they occur. While this may be something that some (but not all) sociologists of risk may have intuitively sensed in the past, there are empirical data from corpus linguistics to suggest now that the semantic prosodies can and do change slightly from one context to another \citeA[p.~177]{hamilton_meanings_2007}.
                        \end{quote}

                        Their dataset included transcribed spoken conversations. This register is remarkably different to that of NYT articles, and examples of risk in these contexts demonstrate this quite clearly (e.g. \emph{Don't don't risk it eh}; \emph{Cos there isn't a risk of going of there}). The key characteristics of these examples (informal lexis, unrecoverable deictic references, low lexical density, etc.) contrast starkly with our examples.

                        Due to the composition of our dataset, we can have little to add to descriptions of risk in casual spoken language, aside from recognising that spoken risk talk is likely to point toward very different, and interesting, results. Though we believe our results may be generalisable to the behaviour of risk in relatively formal written contexts, extended investigation of risk in spoken corpora remains needed.

                        A key finding that received little attention in this earlier linguistic research of risk language is the notion of participants' \emph{agency in risk}. Readily apparent when risk is process is that the kinds of people who risk are typically institutions or humans in positions of power and influence. Actors of risk processes are often states, politicians, or political parties. The \emph{potential harm} being risked is often an abstract concern: \emph{alienating} or \emph{offending} \emph{electorates} or \emph{allies}. In these cases, risk is a process engaged in purposively by Actors who stand to gain something equally abstract. In contrast, when risk functions as a modifier of a participant, the participant is far less powerful: women and children are at-risk of sickness; workers are at risk of injury or death. Here, risk is a quality ascribed to the self. Risky behaviour is not often mentioned. For these people, the potential harm is often recoverable from context, but not outlined within the clause. This distribution was largely consistent throughout our dataset, and will be unpacked through sociological analysis in Chapter N. % note that this may change... 

	\section{Shifting discourse-semantics of risk in the NYT}

                        Some lexicogrammatical and discourse-semantic phenomena have demonstrated consistent shifts over our sampling period. We turn our attention to them now.

                        First, though we noted above that risk as a process involves a different set of participants to risk as a modifier, there are still longitudinal changes within this area. When looking at the \emph{risk of loss}, for example, we can see a general trend toward individual losers, rather than institutional losers. In 1963, the things at risk of loss were macro-level and abstract: athletic funding, market share, vital technology, sympathy in the west, and the like. Later, risked things are more individual assets---life and injury being the two most common. We link this conceptually to neoliberalism:

                        ~\ \todo[inline,color=green!40]{\noindent Sorry Jens, neoliberalism is not easy for me to write up.}
    
                        \subsection{Domains of risk disourse}

                        In terms of the topics in which risk words are deployed, we saw that health risks are very prominent in the more contemporary data samples. Our comparison of \emph{Risk of terror* attack} and \emph{risk of heart attack} demonstrates this preference clearly. This change is indeed a longitudinal one: in 1963 editions, a number of constructions evidence that risk was commonly instantiated with regard to diplomacy, war, international relations, and the like. In their most prominent years, AIDS, Vioxx and Merck comprise over 1.6 per cent of all proper nouns that co-occur with a risk word. This is higher than Clinton, Bush or Obama at their peaks, as well as Soviet Union in 1963\slash 1987 or Europe during the Eurozone crisis in 2011. Moreover, in the years following the AIDS crisis, health risk have increasingly related not to infectious diseases (which require institutional responses), but to kinds of illnesses where the responsibility for prevention falls upon everyday citizens through lifestyle choices, rather than politicians, hospitals, or the FDA. Even in the case of Vioxx, where the risk was created by the premature FDA approval, risk language surrounding Vioxx remained geared toward the risks faced by everyday people. Though Merck and the FDA may be blamed, risk remains a more appropriate frame for discussing the potential for heart attack than it does for discussing the potential harm caused by improper clinical trials or financial interests causing the FDA to approve the medication prematurely.

                        ~\ \todo[inline,color=green!40]{\noindent I'd like to unpack this above, and maybe see if there was a shift toward AIDS as something that needed to be prevented by individual action, rather than govt response ...}

                        As Widdowson (2000) suggests, corpus linguistics often reveals things that are contrary to intuition, and this is certainly the case here. Our expectation of new risk meanings related to terrorism after 9\slash 11 was for the most part not met. Rather than a limitation, this can be treated as an insight in itself: the events and topics that come to mind when we think of risk may not necessarily correspond to the reality of risk language generally. Such is the benefit of corpus linguistic investigation of risk, when compared with previous methodologies employed within the humanities and social sciences to better understand risk.

            \subsection{Implicitness and arguability}

	The most salient theme from the longitudinal mapping of risk is that of implicitness: increasingly common are grammatical constructions where potential harms and risked things are recoverable only from context. Below are three examples in 2012:

        \begin{enumerate}
        \small
            \item \texttt{In 1999, we sold the company, and the next year, we moved to the United States with our two children -- a third was born in 2003 -- so I could pursue my idea of helping low-income, at-risk youth}
            \item \texttt{Carolyn F. Blakely, then a new teacher at the school (who retired last year as the dean of the Honors College that now bears her name at the University of Arkansas at Pine Bluff), remembers Neal as an at-risk kid prone to challenge authority.}
            \item \texttt{Mr. Lane is a sophomore at Lake Academy, an alternative high school for at-risk youths, some of whom take a bus from Chardon High School.}
        \end{enumerate}

        In these cases, what the participant is at-risk \emph{of} is not a specific negative outcome, but an interrelated set of negative outcomes that are more likely to happen to less powerful people in society. Evoked within this cluster is \emph{poverty}, \emph{drug use}, \emph{disease}, \emph{homelessness}, \emph{abuse}, \emph{fatherlessness}, \emph{dropout}, \emph{gang activity}, and the like. In many cases, \emph{at-risk} takes on a euphemistic quality, most obviously as a substitute for \emph{lower-class}, \emph{non-white} or \emph{poor}. Also interesting here is the muddying of the semantic frame: it is both difficult to determine the exact potential harm, and to classify the participant as a \emph{risker}, which seems to imply some agency or comprehension of the risk. More accurately, these participants are \emph{put at risk}---a risk process that itself is on an upward trajectory within our dataset.

        This aligns with the decreasing arguability of risk. Risk in predicator or subject position is increasingly rare, as risk becomes less the nub of propositional meanings. Thus, less and less often is risk a fundamental component in meaning as exchange: in its role within complements and adjuncts, it now more typically plays a supporting role in the provision of information. A ramification of this is that risk becomes an inherent quality of participants in the field of discourse, rather than a process in which participants knowingly or by their own choice choose to engage. This shift is exemplified by the outbound trajectory of \emph{calculated risk}, and its displacement by an uncalculated \emph{potential risk}. In the former, the existence of the risk itself has been acknowledged, and the potential harm\slash reward have been weighed. In the latter, though the situation can be identified as having potentially negative outcomes, these are formless and immeasurable. \emph{Potential risk} is in fact \emph{a risk of risk}. This aligns with the idea that risk (sociological reference) has come to be simply \emph{threat}.

          \subsection{Low-risk, moderate-risk, high-risk}

                During the first years of AIDS, people could be classed according to low, moderate and high-risk groups. Here we have basic quantification of levels of risk. This stands in contrast to the \emph{at-risk} construction discussed above. Of these modifiers, only \emph{low-risk} emerges as an increasingly frequent form. This is also interesting, as it points to a broadening of the semantic scope of risk to include situations where risk remains present: \emph{low-resolution image} does not point toward the increased prominence of low resolution images, but more to the prominence of resolution as thing that meanings are made about. In the same way, the inward trajectory of \emph{low-risk things} does not point toward a culture of less risk, but toward a culture where even things that do not have risk are characterised by their nature to it. We could not locate existing literature supporting a claim that the salience of a concept may be evidenced not only through \emph{extreme case formulations} \emph{the riskiest, high-risk, very risky}, but through minimisation. Nonetheless, our analysis points to the idea that the increased salience of risk as a concept is in part demonstrated through its instantiation in situations where its significance is claimed to be negligible or banal.

            \subsection{Risk as modifier}

            Risk occurs within many different modifier positions:

            \begin{figure}[h!]
            \small
            \centering
            \begin{table}
            \begin{tabular}{|l|l|}
            \hline
            \textbf{Modifier type}       & \textbf{Example}          \\ \hline
            Adjectival pre-head & \emph{a risky move}     \\ \hline
            Post-head           & \emph{A person at risk} \\ \hline
            pre-head nominal    & \emph{risk management}  \\ \hline
            Adverbial           & \emph{to riskily act}   \\ \hline
            Circumstance head   & \emph{to be at risk }   \\ \hline
            \end{tabular}
            \end{table}
            \end{figure}

            Of these, pre-head nominal types are rising, and adjectival pre-head types are falling. From these shifts, we can surmise some sociological insight related to arguability (as conceptualised by SFL). In the increasing frequency of pre-head nominal modifiers (\emph{risk management, risk arbitage, risk factor, risk insurance}), we can see increased social significance of risk as a concept through the evolution of specific jobs whose central concern is risk. Pre-head nominal modification is an indication of codification of a concept: such constructions must be culturally recognised constellations of meaning. In comparison,  adjectives attach to head nouns relatively freely in English. Cultural recognition of the adjective-noun combination is not a prerequisite for meaning to be understood.

            \subsection{Arguability}

            Longitudinal change in the arguability of risk words is consistent. In earlier editions, risk words more commonly occupy more arguable roles, according to systemic functional grammar. In later editions, risk more commonly occurs in heavily dependent positions. Less often does a risk word form the central component being discussed; more often, it exists as a modifier of one of these components, or as a part of a supporting, subordinate clause.

            We are limited in our ability to interpret this result. Little has been written about the relationship between dependency grammars and SFL. As dependencies are inherently functional-semantic, rather than generative-grammatical, dependency is perhaps the most useful \endnote{Current systems for automatic systemic functional annotation tend to rely on dependencies generated with Stanford CoreNLP} mainstream grammar for learning about the semantic behaviour of a given word. That said, though functional categories provided by Stanford CoreNLP's dependency parser overlap in many respects with categories in the Mood system of SFL, there are still mismatches, or shortcomings. Most critically, dependency grammar conflates interpersonal, experiential and textual systems, while SFL demands three separate parses. As discussed earlier, the systemic-functional conceptualisation of subjecthood is threefold, whereas CoreNLP simply nominates the interpersonal subject.

            Due to the availability of nuanced querying languages for phrase structure grammar annotation, our investigation leaned toward grammatical structure annotation over dependency grammar. This is despite a problematic relationship between functional and phrase structure grammars. Given that interesting preliminary findings were unearthed by querying dependency information, we conclude that further exploitation of dependency annotation for the purpose of risk language analysis appears to be a promising area for further analysis.

	\section{Sociological perspectives}


            ~\ \todo[inline,color=green!40]{\noindent Below this point is not particularly readable, sorry.}
            

	The task that remains is to connect observed shifts to their temporal context. In terms of the annual subcorpora, this was by no means a clear-cut task.

	~\ \todo[inline,color=green!40]{\noindent Take it from here, Jens.}

	When focussing on the subcorpora of economic, health and political risks, linguistic reactions to real-world events were much easier to locate. We concluded that further investigation of risk would do well to focus on risk as instantiated within texts sharing a semantic field. 

            Our investigation of topic subcorpora was limited by scope. That said, the open-source tools we have developed for interrogating corpora for discourse analysis could easily be put to use in an investigation of a topic subcorpus.

	~\ \todo[inline,color=green!40]{\noindent Mapping events to risk instantiation in the topic subcorpora here}
	
	We found little evidence that health crises resulted in increased frequency of risk in articles centred on economics or politics. This seems to suggest that while real-world events influence the instantiation of the risk semantic, this instantiation remains more or less limited to the field(s) of discourse to which the real-world event is most related.

	A final point of interest is that adjectival risk words behaved largely contrary to expectations. Adjectival risks as modifiers of participants appear to be decreasing in frequency. Furthermore, though there is a very large variety of adjectival risks, this variety does not seem to be expanding.

	Perhaps in this finding there is some evidence for the Risk Society thesis, in that the ways in which risk can characterise a situation were more or fully articulated during high modernity. Though these characterisations continued to be applied today, saturation point may have been reached.

        \emph{What can be concluded from the finding that real world events do not appear to have long-lasting effects on the behaviour of risk words?}

        Ultimately, perhaps we should not be surprised by this finding. Language is a system that must be resilient against such influences: if single events caused meaningful changes in the lexicogrammatical behaviour of a single word, communication between those aware of and unaware of events would be made more difficult. Accordingly, our suggestion would be that temporary change in the behaviour of a word (as can be seen in spikes in the number of risk words surrounding certain events) are interesting in and of themselves. Moreover, these changes can potentially be measured in pseudo-real-time by mining RSS feeds, using the Twitter API, and so on. Lexicographers could take note of which kinds of events bring about instantiation of a certain word or concept, and create definitions accordingly. Discourse analysts and sociologists could hypothesise the co-occurrence of certain kinds of language with certain kinds of events, and use real-time data to confirm or refute these hypotheses. Cooperative efforts between functional linguistics and sociology, however, are dependent upon a reconciliation of divergent conceptualisations of the relationship between text and context. This issue is elaborated below.

        Predictive applications of Big Data/corpus linguistic methods have already been discussed: \citeA{michel_quantitative_2011} and \citeA{leetaru_culturomics_2011} argue that nuanced mining of large quantities of language can potentially predict civil uprisings such as those seen in the Arab Spring, for example. It must be remembered that these studies have been criticised for their far-reaching conclusions \cite<e.g.>{zimmer_when_????}, and of course that predictive applications of corpus linguistics to date have had the benefit of hindsight. Interpreting peaks and troughs in particular kinds of language is also far from straightforward: increasing numbers of risk words before 9\slash 11 could be interpreted as either a possible predictor of the event or as evidence that the event did not itself cause an increase in the use of risk words. As such, we remain cautiously optimistic about predictive applications. More practically, it seems that such applications are feasible only when there is little delay between text production and text analysis: automated analysis of language circulated via the Web seems a much more sensible starting point for predictive work than static corpora of digitised newspapers and books.

\section{Reconciling sociological and systemic-functional conceptions of text and context}

	Functional linguistic theories such as SFL not only provide a grammar, but also a conceptualisation of the relationship between text and context. In the case of SFL more specifically, the argument is that context is contained within text. Compelling evidence of this is that understanding of the more abstract genre\slash context from which a text is taken can be gained through exposure to lexicogrammar only.

            At issue for sociologists is that this argument rests on a particular operationalisation of the idea of context. The definition according to SFL (though indeed inspired by scholars with significant in sociology, e.g. Malinowski) remains deeply concerned with language. In reality, it is a projection of grammatical phenomena (mood, transitivity, theme) onto the situations and cultures in which texts are produced. Though this has proven a useful heuristic within (critical) discourse analysis, it is in many ways alien to sociological theory, where texts tend to be considered with respect to political, historical and social events and movements, rather than with respect to communicative systems.

	The systemic-functional description of the context of culture contains no references to the effects of current events on language production. While SFL has demonstrated its usefulness without such considerations, this usefulness has been for linguistics. Corpus linguistic applications of SFL have seldom traced the influence of real-world events.

            It is not hard to imagine scenarios where important meanings can be made through absence of references to certain things. In seldom considering both cognitive elements of language production and the influence of specific events, SFL has remained largely unable to conceptualise the notion of meaning made through omission. ...

            Naturally, a current event can influence the likelihood of certain parts of lexicogrammar (the most banal example is in proper nouns, where the appearance of a politician is certain to influence the likelihood of his or her mention in texts).  SFL is predictive in the sense that it can predict that different genres will be more or less likely to involve certain speaker choices. So far, it has not been able to ...

	Like SFL, sociology aims in part to provide a link between text and context. Context, however, is generally not treated as simply a Malinowskian constellation of field, tenor and mode, but also a backdrop of current and past events that inform and shape discourse at the time of its production.

	With these issues in mind, we propose that SFL and general sociological theory are useful partners. SFL provides a means of relating lexicogrammar of texts to discourse-semantic meanings. It can then 

            Key sociological ideas such as reflexive modernity or neoliberalism can be expected to exert influence over texts produced during these movements. Though earlier SFL treats ideology as the most abstract stratum affecting the production of texts, this conceptualisation has been abandoned by a number of current SF linguists.

            Earlier SFL indeed devoted significant energy to exploring the ways in which ideologies such as capitalism are manifested in the content strata of language.

	What often goes unsaid in SF theory is that an additional usefulness of SFL is in its ability to draw a line between the kinds of context (field, tenor and mode) that are embedded within the lexicogrammar of a text and the kinds of context that leave no immediately identifiable trace.

	At this point, sociological theory can fill in the missing parts of the picture. 

	At a level of greater abstraction, functional linguistics and sociological theory can be combined to flesh out the text\slash context relationship. Functional linguistics is concerned with language as a tool to make things happen in the world; sociology can add to the understanding of how culture informs our motivations for making these things happen, for presenting ideas in certain ways, etc.

%\bibliography{../../references/libwin}