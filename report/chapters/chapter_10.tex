%!TEX root = ../risk_report.tex

\chapter{Conclusions}

	Our project has 

	The novel methodology may prove useful to researchers interested in discourse and sociology...




\section{A research agenda for risk discourse research}









	\subsection{More areas of interest}	

	Perhaps the major limitation is on focussing only on risk words to the exclusion of potential

	While necessary here for reasons of scope, the results are indeed 

	\emph{Danger}, \emph{threat}, and \emph{insecurity}. These terms likely 

	Future corpus-based exploration of discourse-semantic change could use methods to map out the behaviour of related concepts, as well as their interrelationships.

\section{Conclusion}

	Of course, instantiation of risk words is linked to real-world events: the beginnings of the AIDS epidemic are accompanied by a spike in health risk discourse; 9\slash 11 appears to be a catalyst for increasing discussion of risks and threats of terror and war. It remains very difficult, however, to fully disentangle the constructive-responsive relationship between real-world events and instantiation of particular concepts in language. Broader ideologies and social movements may indeed be more reliable predictors of linguistic change.

	



%\bibliography{../../references/libwin}