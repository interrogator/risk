%!TEX root = ../risk_book.tex

\chapter{Introduction}

\section{Risk and the social sciences}

\section{New approaches to risk research}

\section{Corpora}

\section{Functional linguistics}

    A number of frameworks exist for connecting lexis and grammar to functional meanings. Notable within risk research has been frame semantics, which has been used to categorise different risk frames and their constituents \cite{fillmore_toward_1992}.

    One such framework is \emph{systemic functional linguistics} \cite<see>{halliday_introduction_2004}.

    We use SFL for three main reasons. First, it is the most detailed functional grammar \cite{eggins_analysing:_2004}: when compared with frame semantics, it provides a more rigorous description of how risk can behave \emph{lexicogrammatically} within a clause. This makes it possible to search parsed texts in nuanced ways. Second, it is a functional-semantic theory, rather than a cognitive-semantic one. While the remarkable achievement of frame semantics is its mapping out of cognitive frames, we are largely unable to operationalise these with our dataset, as we have little information regarding the specific interactants (writers and readers) of the original texts. Moreover, cognitive understandings of text are complicated in situations where the text's author is producing the text within an institutional context, for a readership. Without downplaying the potential importance of cognitivist accounts of risk, we 

     Third, it provides a conceptualisation of the relationship between text and context. A foundational tenet of SFL, and a point of departure from other linguistic theories, is the notion that we can create a description of context based \emph{solely} on the lexicogrammatical content of the text. This is particularly suitable for us, given that our texts arrived to us abstracted from their original contexts. This context was then further obscured through the parsing process.

    



\section{Aim and scope of our project}

Broadly, our project synthesis:

\begin{enumerate}
    \item \emph{Corpus assisted discourse studies} as a \textbf{methodology}
    \item \emph{Systemic functional linguistics} as a \textbf{theory of language}
    \item \emph{Sociological accounts of risk} as \textbf{a set of related assumptions about risk}
\end{enumerate}

Each of these has a limitation of scope:

\begin{enumerate}
    \item A corpus comprised of texts of a single genre and origin
    \item The interpersonal and experiential metafunctions of language
    \item Recent theorising of risk from specific authors (e.g. \emph{x, y, z.})
\end{enumerate}

%\bibliography{../../references/libwin}